%!TEX root=cs112S2015-lab7.tex
% mainfile: cs112S2015-lab7.tex

\input{labspre.tex}

\usepackage[compact]{titlesec}

\begin{document} \MYTITLE{Laboratory Assignment Seven: Doubling Experiments to Assess Time Complexity}
\MYHEADERS{Laboratory Assignment Seven}{Due: March 12, 2015}

\section*{Introduction}

The current module of the course has stressed the importance and purpose of both empirical and analytical evaluations of
algorithm performance. For this laboratory assignment, we will connect these two evaluation strategies as we investigate
the idea of a doubling experiment. First, we will confirm the results that the authors of our textbook reported on the
performance of two methods for string concatenation. Then, you will select an algorithm from those provided, implement
your own doubling experiment framework, conduct your own experiments, collect tables of data, and draw conclusions about
the worst-case time complexity of the chosen algorithm.

\section*{Accessing and Using the String Experiment Framework}

To start this laboratory assignment, you should return to the {\tt cs112S2015-share} Git repository and type the ``{\tt
git pull}'' command in the terminal window.  Now, you should have a {\tt lab7/} directory that you can explore further.
Once again, please make sure that you can find the source code in this new directory and you understand why the
directories in the assignment are structured the way that they are. Next, you should use GVim to study the source code
in the {\tt build.xml} file.  As in the past assignments, when editing a Java program you can type ``{\tt
:Ant compile}'' in your GVim window and it will compile all of the Java classes and save the bytecode in the correct
subdirectories inside of the {\tt bin/} directory.  Please see the instructor if you cannot get this to work.

After you have carefully reviewed the source code for {\tt StringExperiment.java}, you should compile and run this
program, ensuring that you correctly set the {\tt CLASSPATH} as you complete this task. Please run the {\tt
StringExperiment} multiple times and record the data tables that it produces. What trends do you see in this data set?
How do your tables of data compare to the one that the authors present on page 152 of your textbook? Can you clearly
explain why these data values are evident in your data set and the one produced by the textbook's authors?

\section*{Creating and Using Your Own Experimentation Framework}



\section*{Summary of the Required Deliverables}

  This assignment invites you to submit a signed and printed version of the following deliverables:

  \begin{enumerate}
  \itemsep0pt

  \item A sample output from running all doubling experiments in all of their relevant configurations.

  \item A description of all of the configurations supported by your own doubling experiment tool.

  \item The final version of the commented source code for your own doubling experiment tool.

  \item A comprehensive written report that fully explains the results of your experimental studies.

  \item A reflective commentary on the challenges that you faced when implementing your program.

  \item A reflective commentary on the challenges that you faced when conducting the experiments.

  \end{enumerate}

  Along with turning in a printed version of these deliverables, you should ensure that everything is also available in
  the repository that is named according to the convention {\tt cs112S2015-<your user name>}. Please note that students
  in the class are responsible for completing and submitting their own version of this assignment.    While it is
  acceptable for members of this class to have high-level conversations, you should not share source code or full
  command lines with your classmates.  Deliverables that are nearly identical to the work of others will be taken as
  evidence of violating the \mbox{Honor Code}.  Please see the instructor if you have questions about the policies for
  this assignment.

  \end{document}
