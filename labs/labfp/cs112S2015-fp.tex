%!TEX root=cs112S2014-fp.tex 
% mainfile: cs112S2014-fp.tex 

\input{labspre.tex}

\usepackage[compact]{titlesec}

\begin{document} \MYTITLE{Final Project: Automated Analysis of Todo.txt Files}
\MYHEADERS{Final Project}{Due: May 6, 2014 at 5:00 pm}

\section*{Introduction}

Todo.txt is a file format and a suite of tools that supports the management of todo lists and the completion of project
tasks. Originally developed by Gina Trapani, Todo.txt currently includes an ecosystem of tools, such as mobile apps and
desktop applications, that enable individuals and teams to manage projects. In this final assignment, you and your team
members will use Todo.txt to track your progress as you build a tool that can automatically analyze Todo.txt files.

Current Todo.txt tools (e.g., todo.txt-cli, todo.txt-vim, and the Todo.txt Android app) enable a user to input and mark
as completed a wide variety of tasks.  Yet, right now there is no tool that can automatically analyze an existing
Todo.txt file and determine how the time of an individual or a team was spent. You and your team members will implement
a complete tool for the automated analysis of Todo.txt files.  For instance, your tool should be able to determine how
many high-priority tasks you recently completed.  In addition, your tool should be able to ascertain how many tasks were
completed in the different contexts and projects represented in a certain Todo.txt file. 

As you specify, design, implement, test, debug, demonstrate, and release your Todo.txt analysis system, you and your
team members will develop your knowledge and skills in computer science as you learn more about advanced command-line
argument management, the reading and parsing of an input file, and the analysis of frequencies with the hashtable
abstract data type. You will also learn more about the practice of real-world software development as you work in teams
and manage the schedule of your project. Finally, you will hone your writing and speaking skills as you document,
present, and release your working tool for the analysis of Todo.txt files.

\section*{Understanding and Using Todo.txt}

Before you start to specify, design, implement, and test your system, you need to learn more about the Todo.txt
ecosystem that is described at \url{http://todotxt.com/}. You and your team members should study and discuss this site
so that you understand the philosophy behind the Todo.txt system. To gain more experience with using the Todo.txt
system, each of your team members should install and use the todo.txt-vim plugin that is available for download from the
\url{https://github.com/freitass/todo.txt-vim} Web site. To get this plugin fully working, you will need to add a {\tt
Bundle} statement to your {\tt .vimrc} file and then run the {\tt BundleInstall} command in a separate instance of Vim.
To ensure that Vim always enters into todo-mode when you edit a Todo.txt file, you should also add the following line to
your {\tt .vimrc}:

{\tt autocmd BufNewFile,BufRead [Tt]odo.txt set filetype=todo}

Once you have properly configured Vim, you are ready to investigate the use of a Todo.txt file. After creating a
directory in your account called {\tt todo}, you should edit a file called Todo.txt.  Before you start to add content to
this file, you need to learn more about the Todo.txt format, which is described at
\url{https://github.com/ginatrapani/todo.txt-cli/wiki/The-Todo.txt-Format}. Students who would like to study an example
of an existing Todo.txt file can view the one available at \url{http://todotxt.com/todo.txt}. Adding a new task to your
todo list is as simple as starting to type in the Vim text editor.  Moreover, you can mark a task as completed by using
the command {\tt <leader>D} and you can sort all of the tasks in your Todo.txt file by typing {\tt <leader>s}. Before
you finish this phase of the final project, please make sure that all of your team members understand how to use the
priority, context, and project annotations supported by the Todo.txt format.

\section*{Advanced Command-Line Argument Handling}

In past laboratory assignments, you often handled command-line arguments by writing code that would inspect the values
in the {\tt args[]} array that is a parameter to the {\tt main} method.  However, the programs that we have used during
our class and laboratory sessions handled the command-line arguments in a brittle fashion: they required the
command-line arguments to appear in a fixed order and they did not allow for the specification of arguments in a fashion
similar to that used by most programs run in the Linux terminal window. Thankfully, there is a third-party Java library,
called JCommander, that makes it easier to perform advanced argument management. 

Since we want our Todo.txt analysis tool to have a full-featured and easy-to-use command-line argument system, we will
use JCommander for this final project.  You and your team members should learn how to use JCommander by visiting the
following Web sites: \url{http://jcommander.org/} and \url{https://github.com/cbeust/jcommander/}. You can find a
current version of the JCommander JAR file by running the {\tt git pull} command in the {\tt cs112S2014-share/}
repository.  After studying JCommander's user documentation, discussing the library with your team members, and
resolving any questions by talking with the course instructor, your team should consider trying a simple example with
JCommander. Can you display a help menu and recognize the arguments?

\section*{Reading and Parsing Todo.txt Entries}

  As described in Section 1.6 of your textbook, the {\tt java.util.Scanner} class provides methods, like {\tt
  hasNextLine} and {\tt nextLine}, to read in lines from a text file.  Since the Todo.txt file is stored in plain text,
  you and your team members can use an instance of {\tt java.util.Scanner} to input all of its lines into your program.
  Then, your program must break each line of text into its constituent parts.  For instance, suppose that the program
  has just read in the following line from the file:

  (A) @Teaching +112 Write the final project assignment sheet

  The string tokenization process would separately return each part of any string that adheres to the Todo.txt format,
  identifying ``(A)'', ``Teaching'', ``+112'', and the other remaining words. According to Todo.txt terminology, which
  one of these is the priority?  Which is the context?  And, finally, which of these is the project?  To automatically
  answer these questions, you and your team members may decide to investigate the use of a regular expressions and
  methods, like the one called {\tt findInLine}, that are also provided by the {\tt java.util.Scanner}. In addition, you
  may want to look into the use of the Java class called {\tt java.util.StringTokenizer}. Finally, students who want to
  learn more about regular expressions can consult online references such as the following:

  \vspace*{-.25in}
  \begin{itemize}
    \itemsep0in
    \item \url{http://docs.oracle.com/javase/tutorial/essential/regex/}
    \item \url{http://www.vogella.com/tutorials/JavaRegularExpressions/article.html}
  \end{itemize}

\section*{Frequency Analysis with Hashtables}

  Your Todo.txt analysis system should be able to answer a wide variety of questions. For instance, it should be able to
  determine, for each of the priorities inside of the Todo.txt file, how many tasks are assigned a specific priority.
  In addition, the tool should be able to automatically determine, for each of the contexts found in the Todo.txt file,
  how many tasks are associated with a certain context. Finally, the analyzer should determine, for all of the projects
  given in the Todo.txt file, how many of the tasks are connected to each of the projects. Since many other analyses are
  also possible, you and your team member should brainstorm ideas and ensure that your final system implements at least
  five that will enable users to determine how their time is spent.

  Of course, it is important to determine what data structure your program will use to store the specific priorities,
  contexts, and projects.  One candidate that your team should consider is the hashtable, as described in Chapter 9 of
  your textbook and implemented in Java by the {\tt java.util.Hashtable} and {\tt java.util.HashMap} classes. After you
  and your team review the content in this chapter and study online tutorials you will see that the hashtable enables
  you to store key-value pairs.  In the context of this final project, a key may represent, for instance, a task
  priority, and the value would correspond to the number of tasks assigned to that priority. Students who wish to
  learn more about how hashtables can support the counting of frequencies should examine Section 9.2.8 of the textbook
  and discuss the matter further with their team members.

% \section*{Todo.txt Analysis Tasks}

\section*{Summary of the Required Deliverables}

 To complete this final project, your team should evenly divide the work between its members.  In order to support
 collaboration with your team, please create a version control repository that is named according to the convention {\tt
 cs112S2014-<your team number>} and share this repository with each other and the course instructor. By the start of the
 last laboratory session on April 29, 2014 you and your team should be prepared to give a full-featured, public
 demonstration of the current version of your system. Finally, you must finish and release your Todo.txt analyzer and
 submit signed and printed versions of the following deliverables no later than 5 pm on May 6, 2014.

  
  \vspace*{-.05in}
 \begin{enumerate} 
  \itemsep0pt

  \item The Todo.txt file(s) that your team members used to manage the completion of this project

  \item Using examples, a complete description of the features provided by your Todo.txt analysis

  \item Comprehensive user documentation that explains how to run the Todo.txt analyzer

  \item A written explanation of how your program uses scanning, parsing, and hashtables

  \item The output from five separate runs of your final system, demonstrating it's correctness 

  \item A justified statement of the worst-case time complexity for at least two methods 

  \item A privately submitted review of each member of your team, documenting all contributions 

  \item A personal commentary on the challenges that you faced when completing this assignment 
   
  \end{enumerate}
  \vspace*{-.05in}

  Please note that each team is responsible for completing and submitting one version of this assignment.  While it is
  acceptable for members of different teams to have high-level conversations, they should not share source code or
  full command lines with each other.  Deliverables from one team that are nearly identical to the work of another team
  will be taken as evidence of violating the \mbox{Honor Code}.  Please see the instructor if you have questions about
  this policy.

  \end{document}
