%!TEX root=cs112S2015-lab5.tex 
% mainfile: cs112S2015-lab5.tex 

%!TEX root=cs112S2014-lab8.tex
% mainfile: cs112S2014-lab8.tex 
% Typical usage (all UPPERCASE items are optional):
%       \input 580pre
%       \begin{document}
%       \MYTITLE{Title of document, e.g., Lab 1\\Due ...}
%       \MYHEADERS{short title}{other running head, e.g., due date}
%       \PURPOSE{Description of purpose}
%       \SUMMARY{Very short overview of assignment}
%       \DETAILS{Detailed description}
%         \SUBHEAD{if needed} ...
%         \SUBHEAD{if needed} ...
%          ...
%       \HANDIN{What to hand in and how}
%       \begin{checklist}
%       \item ...
%       \end{checklist}
% There is no need to include a "\documentstyle."
% However, there should be an "\end{document}."
%
%===========================================================
\documentclass[11pt,twoside,titlepage]{article}
%%NEED TO ADD epsf!!
\usepackage{threeparttop}
\usepackage{graphicx}
\usepackage{latexsym}
\usepackage{color}
\usepackage{listings}
\usepackage{fancyvrb}
%\usepackage{pgf,pgfarrows,pgfnodes,pgfautomata,pgfheaps,pgfshade}
\usepackage{tikz}
\usepackage[normalem]{ulem}
\tikzset{
    %Define standard arrow tip
%    >=stealth',
    %Define style for boxes
    oval/.style={
           rectangle,
           rounded corners,
           draw=black, very thick,
           text width=6.5em,
           minimum height=2em,
           text centered},
    % Define arrow style
    arr/.style={
           ->,
           thick,
           shorten <=2pt,
           shorten >=2pt,}
}
\usepackage[noend]{algorithmic}
\usepackage[noend]{algorithm}
\newcommand{\bfor}{{\bf for\ }}
\newcommand{\bthen}{{\bf then\ }}
\newcommand{\bwhile}{{\bf while\ }}
\newcommand{\btrue}{{\bf true\ }}
\newcommand{\bfalse}{{\bf false\ }}
\newcommand{\bto}{{\bf to\ }}
\newcommand{\bdo}{{\bf do\ }}
\newcommand{\bif}{{\bf if\ }}
\newcommand{\belse}{{\bf else\ }}
\newcommand{\band}{{\bf and\ }}
\newcommand{\breturn}{{\bf return\ }}
\newcommand{\mod}{{\rm mod}}
\renewcommand{\algorithmiccomment}[1]{$\rhd$ #1}
\newenvironment{checklist}{\par\noindent\hspace{-.25in}{\bf Checklist:}\renewcommand{\labelitemi}{$\Box$}%
\begin{itemize}}{\end{itemize}}
\pagestyle{threepartheadings}
\usepackage{url}
\usepackage{wrapfig}
% removing the standard hyperref to avoid the horrible boxes
%\usepackage{hyperref}
\usepackage[hidelinks]{hyperref}
% added in the dtklogos for the bibtex formatting
\usepackage{dtklogos}
%=========================
% One-inch margins everywhere
%=========================
\setlength{\topmargin}{0in}
\setlength{\textheight}{8.5in}
\setlength{\oddsidemargin}{0in}
\setlength{\evensidemargin}{0in}
\setlength{\textwidth}{6.5in}
%===============================
%===============================
% Macro for document title:
%===============================
\newcommand{\MYTITLE}[1]%
   {\begin{center}
     \begin{center}
     \bf
     CMPSC 112\\Introduction to Computer Science II\\
     Spring 2014
     \medskip
     \end{center}
     \bf
     #1
     \end{center}
}
%================================
% Macro for headings:
%================================
\newcommand{\MYHEADERS}[2]%
   {\lhead{#1}
    \rhead{#2}
    %\immediate\write16{}
    %\immediate\write16{DATE OF HANDOUT?}
    %\read16 to \dateofhandout
    \def \dateofhandout {April 2, 2015}
    \lfoot{\sc Handed out on \dateofhandout}
    %\immediate\write16{}
    %\immediate\write16{HANDOUT NUMBER?}
    %\read16 to\handoutnum
    \def \handoutnum {11}
    \rfoot{Handout \handoutnum}
   }

%================================
% Macro for bold italic:
%================================
\newcommand{\bit}[1]{{\textit{\textbf{#1}}}}

%=========================
% Non-zero paragraph skips.
%=========================
\setlength{\parskip}{1ex}

%=========================
% Create various environments:
%=========================
\newcommand{\PURPOSE}{\par\noindent\hspace{-.25in}{\bf Purpose:\ }}
\newcommand{\SUMMARY}{\par\noindent\hspace{-.25in}{\bf Summary:\ }}
\newcommand{\DETAILS}{\par\noindent\hspace{-.25in}{\bf Details:\ }}
\newcommand{\HANDIN}{\par\noindent\hspace{-.25in}{\bf Hand in:\ }}
\newcommand{\SUBHEAD}[1]{\bigskip\par\noindent\hspace{-.1in}{\sc #1}\\}
%\newenvironment{CHECKLIST}{\begin{itemize}}{\end{itemize}}


\usepackage[compact]{titlesec}

\begin{document} \MYTITLE{Laboratory Assignment Five: Evaluating the Performance of Iteration and Recursion}
\MYHEADERS{Laboratory Assignment Five}{Due: February 19, 2015}

\section*{Introduction}

The Fibonacci sequence includes the numbers in the integer sequence that develops in the following fashion: $0, 1, 1,
2, 3, 5, 8, 13, 21, 34, 55, 89, \ldots$. More formally, we can define the {\em n}-th Fibonacci number, denoted $F_n$,
by the following equation $F_n = F_{n-1} + F_{n-2}$. In this laboratory assignment, we will explore and
extend an implementation of iterative and recursive algorithms for calculating the numbers in the Fibonacci sequence.
Moreover, we will investigate how the bit-depth of the variable that stores the output of the Fibonacci calculator can
influence the correctness of the final result. After learning how to conduct a detailed experimental study of an
algorithm, students will develop and refine their writing skills as they create a comprehensive report of their results.

\section*{Accessing and Using the Fibonacci Benchmarking Framework}

\begin{sloppypar} To start this laboratory assignment, you should return to the {\tt cs112S2015-share} Git repository
  and type the ``{\tt git pull}'' command in the terminal window.  Now, you should have a {\tt lab5/FibonacciBenchmark}
  directory that you can explore further.  Once again, please make sure that you can find the source code in this new
  directory and you understand why the directories in the assignment are structured the way that they are. Next, you
  should use GVim to study the source code in the {\tt build.xml} file.  As in the past assignment, you can type the
  command ``{\tt :Ant compile}'' in GVim and it will compile the three Java classes and save the bytecode in the correct
  subdirectories inside of the {\tt bin/} directory.  Please see the instructor if you cannot get this to work.
\end{sloppypar}

\begin{sloppypar}
As part of this assignment, we will examine the following different implementations: (i) {\tt RecursiveFibonacci} using
{\tt int}, (ii) {\tt RecursiveFibonacci} using {\tt long}, (iii) {\tt IterativeFibonacci} using {\tt int}, and (iv) {\tt
  IterativeFibonacci} using {\tt long}. The {\tt UseFibonacci} class relies on the {\tt Profiler} class is in the {\tt
  profiler.jar} file in the {\tt lib/} directory of the Git repository.  This means that the {\tt UseFibonacci} program
will not run correctly unless you have both the {\tt bin/} directory and the {\tt profiler.jar} archive inside of your
{\tt CLASSPATH} environment variable.
\end{sloppypar}

Try to execute the {\tt UseFibonacci} program for different input values from 0 to 50. You should see the output from
the computation and timing information that is related to the performance of the different algorithms. For example, try
to execute the following command in your terminal window: ``{\tt java edu.allegheny.edu.benchmark.UseFibonacci 10 all}''.

\section*{Adding Additional Features}

  If you study the source code carefully, you will see that the {\tt UseFibonacci} program currently accepts two
  command-line arguments. What are they? As part of this laboratory assignment, you must extend the {\tt UseFibonacci}
  class and the command-line arguments that it can accept. You should add a third command-line argument that corresponds
  to one of the words {\tt int}, {\tt long}, or {\tt all}. After {\tt UseFibonacci} extracts this new command-line
  argument, it should use this additional information to run a specific experiment. One valid execution might be: ``{\tt
    java edu.allegheny.benchmark.Use\-Fibonacci 10 all long}''. This command line would indicate that you should only
  execute the methods within {\tt RecursiveFibonacci} and {\tt IterativeFibonacci} that operate on variables with the
  {\tt long} data type.  You will need to add conditional logic to {\tt UseFibonacci} in order to correctly
  implement this additional feature. Please see the instructor if you have questions about this task.

\section*{Conducting Experiments to Evaluate Efficiency}

Now, you are ready to conduct an experiment to evaluate the performance of the four separate configurations of the
Fibonacci algorithms. You should execute the program with ten different input values from 1 through 50. For each of the
ten input values that you select, you should run the experiment five times and calculate the arithmetic mean. Make sure
that your experiment determines when the {\tt int} primitive type can no longer represent the final answer that is
returned by the method that calculates the Fibonacci number. Your experiment also should identify whether the iterative
or the recursive algorithm is more efficient.  Finally, you should try to determine whether the use of {\tt long} or
{\tt int} yields better performance for the iterative and recursive algorithms. Whenever possible, your report should
explain why these trends are evident in your data sets.

Your report should explain your experimental goals and design by clearly describing the commands that you type and the
order in which you typed them. Your report should also include tables of results that list the running times for each of
the different configurations of {\tt UseFibonacci}.  Make sure that you label the tables and directly reference them in
the text of your report. You must format your report so that it can be easily printed on a sheet of paper that provides
80 characters per single row. Reports will not be accepted unless the tables of data and the paragraphs of analysis are
properly justified.  Please remember that it is not acceptable to simply submit the source code of your program and the
output from running your program. It is very important to write a comprehensive report that identifies the most
noteworthy trends in your data sets.

\section*{Summary of the Required Deliverables}

  This assignment invites you to submit a signed and printed version of the following deliverables: 

    \vspace*{-.1in}
  \begin{enumerate} 
  \itemsep0pt
  \item Using diagram(s), an explanation of how recursion works in the Java programming language.

  \item A short discussion of the different primitive types that are available in the Java language.

  \item The final version of the commented source code for your Fibonacci benchmarking framework.

  \item A comprehensive written report that fully explains the results of your experimental study.

  \item A reflective commentary on the challenges that you faced when implementing the benchmarks.

  \item A reflective commentary on the challenges that you faced when conducting the experiments.
   
  \end{enumerate}
    \vspace*{-.1in}

  Along with turning in a printed version of these deliverables, you should ensure that everything is also available in
  the repository that is named according to the convention {\tt cs112S2015-<your user name>}. Please note that students
  in the class are responsible for completing and submitting their own version of this assignment.    While it is
  acceptable for members of this class to have high-level conversations, you should not share source code or full
  command lines with your classmates.  Deliverables that are nearly identical to the work of others will be taken as
  evidence of violating the \mbox{Honor Code}.  Please see the instructor if you have questions about the policies for
  this assignment.

  \end{document}
