%!TEX root=cs112S2015-lab2.tex
% mainfile: cs112S2015-lab2.tex

\input{labspre.tex}

\usepackage[compact]{titlesec}

\begin{document}
\MYTITLE{Laboratory Assignment Two: Using Vim as an Integrated Development Environment}
\MYHEADERS{Laboratory Assignment Two}{Due: January 29, 2015}

\section*{Introduction}

Practicing software developers normally use an integrated development environment (IDE) to manage various tasks
associated with the design, implementation, and testing of data structures and algorithms. In this course, we will use
Vim as an IDE.  In this laboratory assignment, you will work to learn about the basic features associated with Vim and
then individually prepare your own tutorial that explains how to use basic Vim commands and plugins to support the
navigation and manipulation of Java source code.  The goal for this assignment is to ensure that you can effectively use
Vim, thus enabling you to focus more on the course concepts and less on the use of the IDE.

\section*{Using Runtime Configuration Files}


% After saving this file to the root of your home directory, you should decompress it using the {\tt tar} command in your terminal
% window.  Now, please restart Vim.  What other features have now been added to Vim?  To learn more about how I have configured Vim
% for use in Computer Science 290 Fall 2013, you should study the VimScript in the {\tt .vimrc} and {\tt .gvimrc} files.  Make sure
% that you and your team members understand these configuration files.

\section*{Learning the Basics of Vim}

Before you start to complete the remainder of this laboratory assignment, you may want to review some of the reasons why people
like to use the Vim text editor, as explained at \url{http://usevim.com/2012/10/26/why-vim/}.  When you are finished learning about
some of the reasons behind using Vim, you can return to the GVim window that should still have the source code of a Java program
in it. Using your own program and ultimately writing your own tutorial, you should work with your team members to identify,
learn, and document some of the basic features that are offered by Vim.  For instance, make sure that you know how to perform the
following actions.

\begin{enumerate}

	\item Open, close, and save files in windows or tabs

	\item Move to the beginning and end of a file

	\item Navigate to specific lines and columns within a file

	\item Enter and exit normal mode

	\item Enter and exit insert and append mode

	\item Select line(s) of text in visual mode

	\item Copy, paste, and delete lines of text

	\item Undo the result of a previous command

	\item Search for and replace specific words in a file

	\item Any additional features that you deem to be useful

\end{enumerate}

Since we will be using Vim throughout the semester, please make sure that you can easily invoke all of the editor's most
important commands.  You should take notes and screenshots to demonstrate that you understand how to use basic
Vim commands. As you explore how to use Vim, you should prepare content for a tutorial explaining all of the
aforementioned tasks.

\section*{Using Plugins to Extend Vim}

We will use a variety of Vim plugins to ensure that Vim can operate as a full-fledged integrated development environment when you
complete the laboratory assignments and the final project.  In this phase of the assignment, you are responsible for learning how
to use all of the plugins in the following list.  To start learning more about these plugins, load the source code of your {\tt
  .vimrc} file into GVim and search for a line that starts with the command {\tt Bundle}.  For each of the plugins that you
are required to investigate, you can visit the associated Web site, as listed in the {\tt .vimrc}.

Next, you should run the {\tt :BundleInstall} command in GVim.  After all of the plugins are correctly installed, your enhanced
version of Vim should have many new features! To access these features, you should quit GVim and then run the program again to
view the same Java program that you were previously editing.  Now, you should prepare a tutorial that explains the inputs,
outputs, and behavior of the key features offered by each of the following plugins.

\begin{enumerate}

  \item Ctrl-P
  \item Fugitive
  \item MatchIt
  \item NERDTree
  \item Sneak
  \item TComment
  \item Tagbar

\end{enumerate}

For instance, when learning how to use the Ctrl-P plugin, you should press the key combination {\tt <ctrl-p>} and then use the
interface to navigate the file system and load in new files.  Alternatively, you can press {\tt <F11>} and browse the file system
and load files with the NERDTree plugin.

\section*{Summary of the Required Deliverables}

This assignment invites you to submit one printed version of a tutorial that contains:

\begin{enumerate}

  \item A commentary on how Vim uses runtime configuration files

  \item A full-featured description of the basic features associated with the Vim text editor

  \item A complete introduction to the use of the Vim aforementioned plugins

\end{enumerate}

Along with turning in a printed version of your tutorial, you should ensure that your document is also available in the repository
that is named according to the convention {\tt cs112S2014-<your user name>}. Please note that students in the class are
responsible for completing and submitting their own version of this assignment.  However, you also will be assigned to work in a
team that is tasked with ensuring that all of its members are able to complete each step of the assignment.  Team members should
make themselves available to each other to answer questions and resolve any problems that develop during the laboratory session.
While it is acceptable for members of a team to have high-level conversations, you should not share source code or full command
lines with your team members. To ensure that you can communicate effectively, members of each team should sit next to each other
in the room.  Please see the instructor if you have questions about this policy.

\end{document}
