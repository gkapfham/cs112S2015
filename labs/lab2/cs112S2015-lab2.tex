%!TEX root=cs112S2015-lab2.tex
% mainfile: cs112S2015-lab2.tex

%!TEX root=cs112S2014-lab8.tex
% mainfile: cs112S2014-lab8.tex 
% Typical usage (all UPPERCASE items are optional):
%       \input 580pre
%       \begin{document}
%       \MYTITLE{Title of document, e.g., Lab 1\\Due ...}
%       \MYHEADERS{short title}{other running head, e.g., due date}
%       \PURPOSE{Description of purpose}
%       \SUMMARY{Very short overview of assignment}
%       \DETAILS{Detailed description}
%         \SUBHEAD{if needed} ...
%         \SUBHEAD{if needed} ...
%          ...
%       \HANDIN{What to hand in and how}
%       \begin{checklist}
%       \item ...
%       \end{checklist}
% There is no need to include a "\documentstyle."
% However, there should be an "\end{document}."
%
%===========================================================
\documentclass[11pt,twoside,titlepage]{article}
%%NEED TO ADD epsf!!
\usepackage{threeparttop}
\usepackage{graphicx}
\usepackage{latexsym}
\usepackage{color}
\usepackage{listings}
\usepackage{fancyvrb}
%\usepackage{pgf,pgfarrows,pgfnodes,pgfautomata,pgfheaps,pgfshade}
\usepackage{tikz}
\usepackage[normalem]{ulem}
\tikzset{
    %Define standard arrow tip
%    >=stealth',
    %Define style for boxes
    oval/.style={
           rectangle,
           rounded corners,
           draw=black, very thick,
           text width=6.5em,
           minimum height=2em,
           text centered},
    % Define arrow style
    arr/.style={
           ->,
           thick,
           shorten <=2pt,
           shorten >=2pt,}
}
\usepackage[noend]{algorithmic}
\usepackage[noend]{algorithm}
\newcommand{\bfor}{{\bf for\ }}
\newcommand{\bthen}{{\bf then\ }}
\newcommand{\bwhile}{{\bf while\ }}
\newcommand{\btrue}{{\bf true\ }}
\newcommand{\bfalse}{{\bf false\ }}
\newcommand{\bto}{{\bf to\ }}
\newcommand{\bdo}{{\bf do\ }}
\newcommand{\bif}{{\bf if\ }}
\newcommand{\belse}{{\bf else\ }}
\newcommand{\band}{{\bf and\ }}
\newcommand{\breturn}{{\bf return\ }}
\newcommand{\mod}{{\rm mod}}
\renewcommand{\algorithmiccomment}[1]{$\rhd$ #1}
\newenvironment{checklist}{\par\noindent\hspace{-.25in}{\bf Checklist:}\renewcommand{\labelitemi}{$\Box$}%
\begin{itemize}}{\end{itemize}}
\pagestyle{threepartheadings}
\usepackage{url}
\usepackage{wrapfig}
% removing the standard hyperref to avoid the horrible boxes
%\usepackage{hyperref}
\usepackage[hidelinks]{hyperref}
% added in the dtklogos for the bibtex formatting
\usepackage{dtklogos}
%=========================
% One-inch margins everywhere
%=========================
\setlength{\topmargin}{0in}
\setlength{\textheight}{8.5in}
\setlength{\oddsidemargin}{0in}
\setlength{\evensidemargin}{0in}
\setlength{\textwidth}{6.5in}
%===============================
%===============================
% Macro for document title:
%===============================
\newcommand{\MYTITLE}[1]%
   {\begin{center}
     \begin{center}
     \bf
     CMPSC 112\\Introduction to Computer Science II\\
     Spring 2014
     \medskip
     \end{center}
     \bf
     #1
     \end{center}
}
%================================
% Macro for headings:
%================================
\newcommand{\MYHEADERS}[2]%
   {\lhead{#1}
    \rhead{#2}
    %\immediate\write16{}
    %\immediate\write16{DATE OF HANDOUT?}
    %\read16 to \dateofhandout
    \def \dateofhandout {April 2, 2015}
    \lfoot{\sc Handed out on \dateofhandout}
    %\immediate\write16{}
    %\immediate\write16{HANDOUT NUMBER?}
    %\read16 to\handoutnum
    \def \handoutnum {11}
    \rfoot{Handout \handoutnum}
   }

%================================
% Macro for bold italic:
%================================
\newcommand{\bit}[1]{{\textit{\textbf{#1}}}}

%=========================
% Non-zero paragraph skips.
%=========================
\setlength{\parskip}{1ex}

%=========================
% Create various environments:
%=========================
\newcommand{\PURPOSE}{\par\noindent\hspace{-.25in}{\bf Purpose:\ }}
\newcommand{\SUMMARY}{\par\noindent\hspace{-.25in}{\bf Summary:\ }}
\newcommand{\DETAILS}{\par\noindent\hspace{-.25in}{\bf Details:\ }}
\newcommand{\HANDIN}{\par\noindent\hspace{-.25in}{\bf Hand in:\ }}
\newcommand{\SUBHEAD}[1]{\bigskip\par\noindent\hspace{-.1in}{\sc #1}\\}
%\newenvironment{CHECKLIST}{\begin{itemize}}{\end{itemize}}


\usepackage[compact]{titlesec}

\begin{document}
\MYTITLE{Laboratory Assignment Two: Using Vim as an Integrated Development Environment}
\MYHEADERS{Laboratory Assignment Two}{Due: January 29, 2015}

\section*{Introduction}

Practicing software developers normally use an integrated development environment (IDE) to manage various tasks
associated with the design, implementation, and testing of data structures and algorithms. In this course, we will use
Vim as an IDE.  In this laboratory assignment, you will work to learn about the basic features associated with Vim and
then individually prepare your own tutorial that explains how to use basic Vim commands and plugins to support the
navigation and manipulation of Java source code.  The goal for this assignment is to ensure that you can effectively use
Vim, thus enabling you to focus more on the course concepts and less on the use of the IDE.

% \section*{Using Runtime Configuration Files}

% After saving this file to the root of your home directory, you should decompress it using the {\tt tar} command in your terminal
% window.  Now, please restart Vim.  What other features have now been added to Vim?  To learn more about how I have configured Vim
% for use in Computer Science 290 Fall 2013, you should study the VimScript in the {\tt .vimrc} and {\tt .gvimrc} files.  Make sure
% that you and your team members understand these configuration files.

\section*{Learning the Basics of Vim}

Before you start to complete the remainder of this laboratory assignment, you may want to review some of the reasons why
people like to use the Vim text editor, as explained at \url{http://usevim.com/2012/10/26/why-vim/}.  When you are
finished learning about some of the reasons behind using Vim, you can open a GVim window and load in the source code of
a Java program. Using this code and ultimately writing your own tutorial, you should identify, learn, and document
some of the basic features that are offered by Vim.  For instance, make sure that you know how to perform the following
actions. You must be able to accomplish these tasks without using the mouse; this will both allow you
to use Vim in a terminal and to write Java programs more rapidly.

\begin{enumerate}
  \itemsep 0em

        \item Open, close, and save files in windows or tabs
        \item Move to the beginning and end of a file
        \item Navigate to specific lines and columns within a file
        \item Enter and exit normal mode
        \item Enter and exit insert and append mode
        \item Select line(s) of text in visual mode
        \item Copy, paste, and delete lines of text
        \item Undo the result of a previous command
        \item Search for and replace specific words in a file
        \item Any additional features that you deem to be useful

\end{enumerate}

Since we will be using Vim throughout the semester, please make sure that you can easily invoke all of the editor's most
important commands.  You should take notes and screenshots to demonstrate that you understand how to use basic
Vim commands. As you explore how to use Vim, you should prepare content for a tutorial explaining all of the
aforementioned tasks.

\section*{Using Plugins to Extend Vim}

We will use a variety of Vim plugins to ensure that Vim can operate as a full-fledged integrated development environment
when you complete the laboratory assignments and the final project.  In this phase of the assignment, you are
responsible for learning how to use all of the plugins in the following list.  If you have not already done so, please
make sure that you install the spf13-vim distribution; you can learn more about this approach to enhancing Vim by
visiting \url{http://vim.spf13.com/}.  Once you have installed spf13-vim and you are sure that all of its associated
plugins are installed correctly, you should prepare a tutorial that explains the inputs, outputs, and behavior of the
key features offered by each of the following plugins.

\begin{enumerate}
  \itemsep 0em
  \item Ctrl-P
  \item Fugitive
  \item NERDTree
  \item NERDCommenter
  \item EasyMotion
  \item Tabularize
  \item Tagbar
\end{enumerate}

For instance, when learning how to use the Ctrl-P plugin, you should press the key combination {\tt <ctrl-p>} and then
use the interface to navigate the file system and load in new files.  Alternatively, you can press {\tt <ctrl-E>} and
browse the file system and load files with the NERDTree plugin. Please make sure that you understand, in detail, how to
use each of the aforementioned plugins.  That is, even if you already have an ``tried and true'' method for
accomplishing a certain task in GVim, you should push yourself to explore the features of these plugins and determine if
you can become a more efficient implementor of programs in GVim.

\section*{Summary of the Required Deliverables}

This assignment invites you to submit one printed version of a tutorial that contains:

\begin{enumerate}

  \item A commentary on how Vim uses runtime configuration files

  \item A full-featured description of the basic features associated with the Vim text editor

  \item A complete introduction to the use of the Vim aforementioned plugins

\end{enumerate}

% Along with turning in a printed version of your tutorial, you should ensure that your document is also available in the repository
% that is named according to the convention {\tt cs112S2014-<your user name>}. Please note that students in the class are
% responsible for completing and submitting their own version of this assignment.  However, you also will be assigned to work in a
% team that is tasked with ensuring that all of its members are able to complete each step of the assignment.  Team members should
% make themselves available to each other to answer questions and resolve any problems that develop during the laboratory session.
% While it is acceptable for members of a team to have high-level conversations, you should not share source code or full command
% lines with your team members. To ensure that you can communicate effectively, members of each team should sit next to each other
% in the room.  Please see the instructor if you have questions about this policy.

Before you turn in this assignment, you also must ensure that the course instructor has read access to your Bitbucket
repository that is named according to the convention {\tt cs112S2015-<your user name>}. Please note that each student in
the class is responsible for completing and submitting their own version of this assignment. While it is acceptable for
members of this class to have high-level conversations, you should not share source code or full command lines with your
classmates.  That is, it is necessary to distinguish carefully between the student who discusses the principles
underlying a problem with others and the student who produces assignments that are identical to, or merely variations
on, someone else's work.  Source code or output that is largely similar to other submissions or to online material will
be judged as evidence of violating the Honor Code. Please see the course instructor if you have questions about the policies
for this laboratory assignment.

\end{document}
